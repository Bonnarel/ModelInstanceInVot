\begin {itemize}
  \item Shy annotation: model annotations come in a workflow that works very well for years, this is why the first requirement is to not break any existing stakeholder.
  \begin {itemize}
    \item Annotation must not alter the VOTable content.
    \item Annotation blocks must be located in a way that it can easily be skipped by model-unaware clients.
    \item The vocabulary in the annotation name-space must not overlap the VOTable elements (names or attributes)    
    \item The annotation syntax must be able to inform the client about the status of the annotation process.
  \end {itemize}
  
  \item Schema and validation:
  \begin {itemize}
    \item The annotation schema must be independent from the VOTable schema.
    \item The evolution of the annotation schema must not impact the VOTable schema.
    \item The evolution of the VOTable schema must not impact the annotation schema.
    \item The annotation syntax must be validated with regular tools.
    \item The validation does not check whether references can be resolved.  Handling such inconsistencies is in the charge of the client
  \end {itemize}
  
  \item Model agnostic:
  \begin {itemize}
    \item The annotation syntax must be able to map data on any VODML compliant model
    \item The annotation syntax must allow clients to use their own strategy to consume mapped data:
      \begin {itemize}
        \item just ignore it
        \item just pick some elements of interest 
        \item just pick model meta-data and process the stream of data rows as usual
        \item pick whole model instances
      \end {itemize}
  \end {itemize}
  
\end {itemize}
