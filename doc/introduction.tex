
The first purpose of a model is to provide, for a particular domain, a formal description of the relevant quantities and of the way they are connected together .
This documentary role facilitates the communication between the stack-holders and thus the design of interoperability protocols. 

At data level, interoperability consists in arranging searched data in a way that a client can understand them without taking care of their origin. 
So that, the same code can process and compare data coming from different sources.  
This can be achieved when the data has the same representation, therefore when it is mapped on the same model.

VOTables cannot take this feature in charge they are containers \citep{2019ivoa.spec.1021O}. 
The VOTable schema allows a fine grain description of the fields but it cannot show how data is mapped on a given model or whether it match any model at all. 
This is not an issue for simple protocol responses (ref) because the VOTable structure is defined by the protocol itself. 
This is might be a big issue for VOTables containing native data such as Vizier  or TAP query responses however.

The challenge here is to bind native data with a model in a way that a model-aware software can see it as 
model instances while maintaining the possibility to access them in their original shape.

This is partially done with UTypes which may connect FIELDs or PARAMs with model leaves in the case of tree-like models. 
Unfortunately, there is nos unique  way to build and parse UTypes in the context of complex models. 
UTypes are suited patterns where e.g the same class is used in different location of the model or when the model contains loops. 
UTypes do not allow cross-table references either.

The landscape has dramatically changed in 2016 when VODML \citep{2018ivoa.spec.0910L} became a recommendation. 
VODML is a meta-model that gives a standard way to express VO models and to make them machine-readable.
In VODML, model leaves are no longer identified by a simple string like UTypes do but by a certain role played in a given location in the model hierarchy.
As a consequence any annotation mechanism based on VODML and preserving the model hierarchy will be able  to provide a data representation faith to the model.

The main concept of this standard is to insert in VOTable resources an XML block complying with the 
model structures and containing references to the actual data.
These blocks are designed in such a way that a model-aware client only has to make a copy of that structure and to resolve the references  
to build model instances. More generic model-unaware clients can just ignore the mapping blocks. 


