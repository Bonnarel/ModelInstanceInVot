VO-DML structured types are annotated by using the INSTANCE
element. Note that there is no difference, from a schema point of view,
between **ObjectType**s and **DataType**.

\begin{table}[!htbp]
\small
\centering
\begin{tabulary}{\linewidth}{|c|J|}       
       \hline 
            \textbf{Attribute} & 
            \textbf {Role}\\
       \hline         \hline  
            @ID & 
            ID of the mapping element  \\
        \hline 
            @dmrole & 
            Role of the instance in the DM \\
        \hline 
            @dmtype & 
            Class name \\
        \hline 
     \end{tabulary}
     \caption{\texttt{INSTANCE} attributes} 
     \label{tbl:instance-att}
 \end{table}

\begin{table}[!htbp]
\small
\centering
\begin{tabulary}{\linewidth}{|c|c|c|J|}
    \hline 
        \textbf{@ID} &
        \textbf{@dmrole} &
        \textbf{@dmtype} &
        \textbf{Pattern}\\
    \hline      \hline  
        MAND &           
        NO &           
        MAND &           
        The instance, usually located in \texttt{GLOBALS}, has no role. It can be referenced by a \texttt{REFERENCE}  \\
    \hline   
        OPT &           
        MAND &           
        MAND &           
        The element maps a instance playing a role in the model. @ID can also be set in that case. \\
   \hline 
\end{tabulary}
     \caption{Valid attribute patterns for  \texttt{INSTANCE}} 
     \label{tbl:instance-pattern}
 \end{table}


\begin{table}[!htbp]
\small
\centering
\begin{tabulary}{\linewidth}{|c |c |c|J|}
    \hline 
        \textbf{Element} &
        \textbf{Position} &
        \textbf{Cardinality} &
        \\
    \hline      \hline  
        \texttt{REFERENCE}  &        
        Any &           
        0-* &
         Object attribute as a reference to either a class or a collection instance.\\
    \hline    
        \texttt{INSTANCE} &           
        Any &           
        0-* &
         Object attribute as a class instance. \\
    \hline    
        \texttt{ATTRIBUTE} &           
        Any &           
        0-* &
       Object attribute as a simple attribute. \\
    \hline    
        \texttt{COLLECTION} &           
        Any &           
        0-* &
         Object attribute  as a collection.\\
    \hline 
\end{tabulary}
     \caption{Allowed children for \texttt{INSTANCE}} 
     \label{tbl:instance-chilren}
 \end{table}