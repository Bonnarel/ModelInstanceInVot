A VOTable can provide serializations for an arbitrary number of data model
types. In order to declare which models are represented in the file, data
providers must declare them through the MODEL elements.
Only models that are used in the file must be declared. A model is
used if at least one element in the mapping block refer to it. In other terms, only models that define vodml-ids used in the
annotation must be declared.

\begin{lstlisting}[frame=single,caption={Example MODEL mapping block},style=XML,basicstyle=\tiny]
<dm-mapping:VODML>
  <dm-mapping:MODEL name="sample-ext"
                     url="https://www.myorg.net/models/SampleExt-v1.0.vo-dml.xml" />
  <dm-mapping:MODEL name="sample" url="https://www.ivoa.net/xml/DNE/Sample-v1.0.vo-dml.xml" />
  <dm-mapping:MODEL name="ivoa"   url="https://www.ivoa.net/xml/VODML/IVOA-v1.vo-dml.xml" />
</dm-mapping:VODML>
\end{lstlisting}

\begin{table}[!htbp]
  \small
  \centering
  \begin{tabulary}{\linewidth}{|c|J|}       
    \hline 
         \textbf{Attribute} & 
         \textbf {Role}\\
    \hline
    \hline  
         @name  & 
         Name of the mapped model as declared in the VODML/XML model serialization.  This attribute MUST not be empty and forms the prefix used in dmrole/dmtype tags of elements from that model.  \\
    \hline 
         @url & 
         URL to the vo-dml serialization of the model. If present, this attribute MUST not be empty.\\
    \hline 
  \end{tabulary}
  \caption{\texttt{MODEL} attributes} 
  \label{tbl:model-att}
\end{table}


\begin{table}[!htbp]
  \small
  \centering
  \begin{tabulary}{\linewidth}{|c |c |J|}
    \hline 
        \textbf{@name} &
        \textbf{@url} &
        \textbf{Pattern}\\
    \hline      \hline  
        MAND &           
        OPT &           
        Unique attribute pattern supported by \texttt{MODEL}\\
    \hline 
  \end{tabulary}
  \caption{Valid attribute patterns for  \texttt{MODEL}} 
  \label{tbl:model-pattern}
\end{table}
