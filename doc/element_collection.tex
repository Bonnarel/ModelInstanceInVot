    COLLECTION is a container element.  It is used in different contexts, each allowing a limited subset of elements for its content. 
    
    COLLECTION can be populated either by a static list of items (INSTANCE, ATTRIBUTE,..) or by one JOIN operation. Both  modes cannot coexist.
    %(FB is "operation the right term in this xml context ?)
    
    \begin{enumerate}
    \item{As child of INSTANCE}
      
      The COLLECTION serves as a container for elements with multiplicity $>$ 1.\\
      Examples of usage in this context would be:
      \begin{itemize}
        \item an array attribute
        \item a reference relation with multiplicity $>$ 1
        \item a composition relation with multiplicity $>$ 1
      \end{itemize}
      
    \item{As child of GLOBALS}
          
      The COLLECTION serves as a proxy for TABLE, grouping common INSTANCES for selection by PRIMARY/FOREIGN\_KEY.
     
      Examples of usage in this context would be:
      \begin{itemize}
        \item a set of photometry filters, which apply to various rows of a photometric data table, based on the value of the 'band' column.
        \item a set of Dataset metadata instances, which apply to various rows of a photometric data table, based on the value of the 'id' column.
      \end{itemize}
          
    \item{As child of COLLECTION}
	The collection contains a matrix of  atomic values
        
    \end{enumerate}
   
\begin{lstlisting}[frame=single,caption={Example of COLLECTION child of INSTANCE},style=XML,basicstyle=\tiny]
<dm-mapping:INSTANCE dmtype="model:Thing">
    <dm-mapping:COLLECTION dmrole="model:Thing.elems">
        <dm-mapping:ATTRIBUTE dmtype="model:Foo" value="100" />
        <dm-mapping:ATTRIBUTE dmtype="model:Foo" value="110" />
    </dm-mapping:COLLECTION>
</dm-mapping:INSTANCE>
\end{lstlisting}   

\begin{lstlisting}[frame=single,caption={Example of COLLECTION child of GLOBALS},style=XML,basicstyle=\tiny]
<dm-mapping:GLOBALS>
    <dm-mapping:COLLECTION dmid="_filters" >
        <dm-mapping:INSTANCE dmtype="model:PhotometryFilter" >
            <dm-mapping:PRIMARY_KEY dmtype="ivoa:string" value="RP"/>
            <dm-mapping:ATTRIBUTE dmrole="model:PhotometryFilter.name" dmtype="ivoa:string"
                                    value="GAIA/GAIA2r.Grp"/>
        </dm-mapping:INSTANCE>
        <dm-mapping:INSTANCE dmtype="model:PhotometryFilter" >
            <dm-mapping:PRIMARY_KEY dmtype="ivoa:string" value="BP"/>
            <dm-mapping:ATTRIBUTE dmrole="model:PhotometryFilter.name" dmtype="ivoa:string"
                                    value="GAIA/GAIA2r.Gbp"/>
        </dm-mapping:INSTANCE>
    </dm-mapping:COLLECTION>
<dm-mapping:GLOBALS>
\end{lstlisting}   

\begin{lstlisting}[frame=single,caption={Example of COLLECTION populated with a JOIN},style=XML,basicstyle=\tiny]
<dm-mapping:GLOBALS>
    <dm-mapping:COLLECTION dmrole="_joined_dataj">
        <dm-mapping:JOIN tableref="_someTable" dmref="_extInst"/>
    </dm-mapping:COLLECTION>
<dm-mapping:GLOBALS>
\end{lstlisting}   
%(FB I don't understand this. Is that a natutal join ? Maybe we should say that ?)


\begin{table}[!htbp]
  \small
  \centering
  \begin{tabulary}{\linewidth}{|c|J|}       
    \hline 
         \textbf{Attribute} & 
         \textbf {Role}\\
    \hline
    \hline  
         @dmid & 
         Element datamodel id, MUST be unique within the document.\\
    \hline 
         @dmrole & 
         Role of the \texttt{COLLECTION} in the data model. \\
    \hline 
  \end{tabulary}
  \caption{\texttt{COLLECTION} attributes} 
  \label{tbl:collection-att}
 \end{table}

\begin{table}[!htbp]
  \small
  \centering
  \begin{tabulary}{\linewidth}{|c|c|c|J|}
    \hline 
      \textbf{Context} &
      \textbf{@ID} &
      \textbf{@dmrole} &
      \textbf{Pattern}\\
    \hline      \hline  
      1 &
      OPT & 
      MAND & 
      The element maps a collection playing a role in a modeled \texttt{INSTANCE}.  @dmrole MUST  be not empty.  If present, @dmid MUST  be not empty. \\
    \hline   
      2 &
      MAND & 
      NO & 
      The collection, has no role. MUST have non-empty dmid to reference for ORM selection of contained \texttt{INSTANCE}. This occurs when e.g. the COLLECTION is a GLOBALS child.\\
    \hline 
  \end{tabulary}
  \caption{Valid attribute patterns for \texttt{COLLECTION}} 
  \label{tbl:collection-pattern}
 \end{table}


\begin{table}[!htbp]
  \small
  \centering
  \begin{tabulary}{\linewidth}{|c|c|c|J|}
    \hline 
      \multicolumn{4}{|l|}{\textbf{Context: Child of INSTANCE}} \\
    \hline 
      \textbf{Element} &
      \textbf{Position} &
      \textbf{Occurs}  \\
    \hline
    \hline  
        \texttt{ATTRIBUTE} & 
        Only & 
        0-* & 
        Collection of attributes.\\
    \hline    
        \texttt{REFERENCE} & 
        Only & 
        0-* & 
        Collection of references.\\
    \hline    
        \texttt{INSTANCE} & 
        Any & 
        0-* & 
        Collection of instances.\\
    \hline    
        \texttt{JOIN} & 
        Any & 
        0-1 & 
        Collection populated with a set of joined instances. No other child is supported in this case\\
    \hline    
        \texttt{COLLECTION} & 
        Only & 
        0-* & 
        Collection of collections.\\
    \hline    
    \hline 
      \multicolumn{4}{|l|}{\textbf{Context: Child of GLOBALS}} \\
    \hline 
      \textbf{Element} &
      \textbf{Position} &
      \textbf{Occurs} &  \\
    \hline
    \hline    
        \texttt{INSTANCE} & 
        Only & 
        0-* & 
        Collection of related instances.\\
    \hline    
        \texttt{REFERENCE} & 
        Only & 
        0-* & 
        Collection of related references.\\
    \hline 
  \end{tabulary}
     \caption{Allowed children for \texttt{COLLECTION}} 
     \label{tbl:collection-chilren}
 \end{table}
